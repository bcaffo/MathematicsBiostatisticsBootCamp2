\documentclass[aspectratio=169]{beamer}
\mode<presentation>
%\usetheme{Warsaw}
%\usetheme{Goettingen}
\usetheme{Hannover}
%\useoutertheme{default}

%\useoutertheme{infolines}
\useoutertheme{sidebar}
\usecolortheme{dolphin}

\usepackage{amsmath}
\usepackage{amssymb}
\usepackage{enumerate}

%some bold math symbosl
\newcommand{\Cov}{\mathrm{Cov}}
\newcommand{\Cor}{\mathrm{Cor}}
\newcommand{\Var}{\mathrm{Var}}
\newcommand{\brho}{\boldsymbol{\rho}}
\newcommand{\bSigma}{\boldsymbol{\Sigma}}
\newcommand{\btheta}{\boldsymbol{\theta}}
\newcommand{\bbeta}{\boldsymbol{\beta}}
\newcommand{\bmu}{\boldsymbol{\mu}}
\newcommand{\bW}{\mathbf{W}}
\newcommand{\one}{\mathbf{1}}
\newcommand{\bH}{\mathbf{H}}
\newcommand{\by}{\mathbf{y}}
\newcommand{\bolde}{\mathbf{e}}
\newcommand{\bx}{\mathbf{x}}

\newcommand{\cpp}[1]{\texttt{#1}}

 \title{Mathematical Biostatistics Boot Camp 2: Lecture }
\author{Brian Caffo}
\date{\today}
\institute[Department of Biostatistics]{
  Department of Biostatistics \\
  Johns Hopkins Bloomberg School of Public Health\\
  Johns Hopkins University
}


\begin{document}
\frame{\titlepage}

%\section{Table of contents}
\frame{
  \frametitle{Table of contents}
  \tableofcontents
}

\section{Multiplicity}
\begin{frame}\frametitle{Multiplicity}
\begin{itemize}
\item After rejecting a $\chi^2$ omnibus 
  test you do all pairwise comparisons
\item You conducted a study with 20 outcomes and
  30 different combinations of covariates. You 
  consider significance at all combinations.
\item You compare diseased tissue versus normal tissue
  expression levels for 20$k$ genes
\item You compare rest versus active at 300$k$ voxels
  in an fMRI study
\end{itemize}
\end{frame}

\begin{frame}\frametitle{Multiplicity}
\begin{itemize}
\item Performing two $\alpha$-level tests: \\
  $H_0^1$ versus $H_a^1$ and $H_0^2$ versus $H_a^2$ \\
  $E_1$ Reject $H_0^1$ and  $E_2$ Reject $H_0^2$
  \begin{eqnarray*}
FWE 
&   & P(\mbox{one or more false rej} ~|~ H_0^1, H_0^2) \\
& = & P(E_1 \cup E_2 ~|~ H_0^1, H_0^2) \\
& = & P(E_1 ~|~ H_0^1, H_0^2) + P(E_2 ~|~ H_0^1, H_0^2) \\
& - & P(E_1 \cap E_2 ~|~ H_0^1, H_0^2) \\
& \leq & P(E_1 ~|~ H_0^1, H_0^2) + P(E_2 ~|~ H_0^1, H_0^2) \\
& = & 2\times \alpha
  \end{eqnarray*}
\end{itemize}
Result : The {\bf familywise error rate} for 
$k$ hypotheses tested at level $\alpha$ is 
bounded by $k\alpha$
\end{frame}


\begin{frame}\frametitle{Proof}
$E_i$ - false rejection for test $i$ \\
All probabilities are conditional on all of the nulls being true 
\begin{eqnarray*}
FWE 
& =      & P(\mbox{one or more false rej}) \\
& =      & P(\cup_{i=1}^k E_i) \\
& =      & P\left\{E_1 \cup (\cup_{i=2}^k E_i)\right\} \\
& \leq   & P(E_1) + P(\cup_{i=2}^k E_i)\\
& \vdots & \\
& \leq   & P(E_1) + P(E_2) +\hdots+P(E_k)\\
& =      & k\alpha 
\end{eqnarray*}
\end{frame}

\begin{frame}\frametitle{Other direction}
\begin{itemize}
\item The $FWE$ is no larger than $k\alpha$ where $k$ is the number of tests
\item The $FWE$ is no smaller than $\alpha$ 
$$
P(\cup_{i=1}^k E_i) \geq P(E_1) = \alpha
$$
\item The lower bound is obtained when the $E_i$ are identical
  $E_1 = E_2 = \ldots = E_k$ 
\item {\bf Bonferoni's} tests each individual hypothesis at level $\alpha^* = \alpha / k$
  
  \begin{itemize}
  \item The $FWE$ is no larger than $k \alpha^* = k \alpha / k = \alpha$
  \item The $FWE$ is no smaller than $\alpha / k$
  \end{itemize}
\end{itemize}
\end{frame}


\section{Bonferoni}
\begin{frame}\frametitle{Bonferoni's procedure} 
If $\alpha^*$ is small and the tests are independent, then the
upper bound on the $FWE$ is nearly obtained 

\begin{eqnarray*}
 FWE & = & P(\mbox{one or more false rej}) \\
     & = & 1 - P(\mbox{no false rej}) \\
     & = & 1 - P(\cap_{i=1}^k \bar E_i) \\
     & = & 1 - (1 - \alpha^*)^k \\
     & \approx & 1 - (1 - k\alpha^*)\\
     & = & k \alpha^* = \alpha
\end{eqnarray*}
\end{frame}

\begin{frame}\frametitle{Scratch work} 
Recall the approximation for $\alpha^*$ near 0
$$
\frac{f(\alpha^*) - f(0)}{\alpha^* - 0} \approx f'(0) 
$$
hence
$$
f(\alpha^*) \approx f(0) + \alpha^* f'(0)
$$ \\
In our case $f(\alpha^*) = (1 - \alpha^*)^k$ so $f(0) = 1$ \ \\ \ \\
$f'(\alpha^*) = -k(1 - \alpha^*)^{k-1}$ so $f'(0) = -k$
\ \\ \ \\
Therefore $(1 - \alpha^*)^k \approx 1 - k \alpha^*$
\end{frame}

\begin{frame}\frametitle{Notes}
\begin{itemize}
\item For Bonferoni's procedure $\alpha^* = \alpha / k$ so will be close to 0 for 
  a large number of tests
\item When there are lots of tests that are (close to) independent, 
  the upper bound on the $FWE$ used is appropriate
\item When the test are closely related, then the $FWE$ will be closer to the lower
  bound, and Bonferoni's procedure is conservative
\item Is the familywise error rate always the most appropriate quantity to control for?
\end{itemize}
\end{frame}

\section{FDR}
\begin{frame}\frametitle{FDR}
\begin{itemize}
\item The {\bf false discovery rate} is the proportion of tests that are
  falsely declared significant
\item Controlling the FDR is less conservative than controlling the FWE rate
\item Introduced by Benjamini and Hochberg
\end{itemize}
\end{frame}


\begin{frame}\frametitle{Benjamini and Hochberg procedure}
\begin{enumerate}
\item Order your $k$ p-values, say $p_1 < p_2 < \ldots <p_k$
\item Define $q_i = k p_i / i$
\item Define $F_i = min(q_i,\ldots,q_k)$
\item Reject for all $i$ so that $F_i$ is less than the desired FDR
\end{enumerate}
Note that the $F_i$ are increasing, so you only need to find the largest
one so that $F_i < FDR$
\end{frame} 

\begin{frame}\frametitle{Example}
1st 10 of 50 SNPs (Rosner page 581)\\
\begin{center}
\ttfamily
\begin{tabular}{lllll}
Gene & $i$ & $p_i$ & $q_i=kp_i/i$ & $F_i$ \\ \hline
 30 & 1 & $<$.0001 & .0035 & .0035 \\
 20 & 2 & .011     & .28   & .16   \\
 48 & 3 & .017     & .28   & .16   \\
 50 & 4 & .017     & .22   & .16   \\
  4 & 5 & .018     & .18   & .16   \\
 40 & 6 & .019     & .16   & .16   \\
  7 & 7 & .026     & .18   & .18   \\
 14 & 8 & .034     & .21   & .21   \\
 26 & 9 & .042     & .23   & .23   \\
 47 & 10& .048     & .24   & .24   \\ \hline
\end{tabular}
\normalfont
\end{center}
\end{frame}

\begin{frame}\frametitle{Example}
\begin{itemize}
\item Bonferoni cutoff $.05 / 50 = .001$; only the first Gene is significant
\item For a FDR of $0-15\%$; only the first Gene would be declared significant
\item For a FDR of $16-20\%$, the first $7$ would be significant
\end{itemize}
\end{frame}


\end{document}
