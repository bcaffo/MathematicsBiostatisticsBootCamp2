\documentclass[aspectratio=169]{beamer}
\mode<presentation>
%\usetheme{Warsaw}
%\usetheme{Goettingen}
\usetheme{Hannover}
%\useoutertheme{default}

%\useoutertheme{infolines}
\useoutertheme{sidebar}
\usecolortheme{dolphin}

\usepackage{amsmath}
\usepackage{amssymb}
\usepackage{enumerate}

%some bold math symbosl
\newcommand{\Cov}{\mathrm{Cov}}
\newcommand{\Cor}{\mathrm{Cor}}
\newcommand{\Var}{\mathrm{Var}}
\newcommand{\brho}{\boldsymbol{\rho}}
\newcommand{\bSigma}{\boldsymbol{\Sigma}}
\newcommand{\btheta}{\boldsymbol{\theta}}
\newcommand{\bbeta}{\boldsymbol{\beta}}
\newcommand{\bmu}{\boldsymbol{\mu}}
\newcommand{\bW}{\mathbf{W}}
\newcommand{\one}{\mathbf{1}}
\newcommand{\bH}{\mathbf{H}}
\newcommand{\by}{\mathbf{y}}
\newcommand{\bolde}{\mathbf{e}}
\newcommand{\bx}{\mathbf{x}}

\newcommand{\cpp}[1]{\texttt{#1}}

\title{Mathematical Biostatistics Boot Camp 2: 
Lecture 9, Simpson's Paradox and Confounding}
\author{Brian Caffo}
\date{\today}
\institute[Department of Biostatistics]{
  Department of Biostatistics \\
  Johns Hopkins Bloomberg School of Public Health\\
  Johns Hopkins University
}


\begin{document}
\frame{\titlepage}

%\section{Table of contents}
\frame{
  \frametitle{Table of contents}
  \tableofcontents
}

\section{Simpson's paradox}
\begin{frame}\frametitle{Simpson's (perceived) paradox}
\begin{center}
\ttfamily
  \begin{tabular}{llccr}
         &             & \multicolumn{2}{c}{Death penalty} &     \\ \cline{3-4}
Victim   & Defendant   &  yes  & no                        & \% yes \\ \hline
White & White & 53 & 414 & 11.3 \\
      & Black & 11 & 37  & 22.9 \\
Black & White & 0  & 16  &  0.0 \\
      & Black & 4  & 139 &  2.8 \\ \hline
      & White & 53 & 430 & 11.0 \\
      & Black & 15 & 176 &  7.9 \\ \hline
White &       & 64 & 451 & 12.4 \\
Black &       &  4 & 155 &  2.5 \\ \hline
  \end{tabular}
\normalfont
\end{center}
~\footnote{From Agresti, Categorical Data Analysis, second edition}
\end{frame}

\begin{frame}\frametitle{Discussion}
\begin{itemize}
\item Marginally, white defendants received the death penalty a greater percentage of 
  time than black defendants
\item Across white and black victims, black defendant's received the
  death penalty a greater percentage of time than white defendants
\item Simpson's paradox refers to the fact that marginal and conditional associations
  can be opposing
\item The death penalty was enacted more often for the murder of a white victim than
  a black victim. Whites tend to kill whites, hence the larger marginal association.
\end{itemize}
\end{frame}

\section{More examples}
\begin{frame}\frametitle{Example}
\begin{itemize}
\item Wikipedia's entry on Simpson's paradox gives an example
  comparing two player's batting averages
  \begin{center}
    \ttfamily
    \small
    \begin{tabular}{llll} 
               & First             & Second             & Whole  \\
               & Half              & Half               & Season \\ \hline
      Player 1 & $~4 / 10 ~~(.40)$ ~~& $25 / 100 ~(.25)$ ~~& $29 / 110 ~(.26)$ \\
      Plater 2 & $35 / 100 ~(.35)$ ~~& $~2 / 10~ ~(.20)$ ~~& $37 / 110 ~(.34)$ \\ \hline 
    \end{tabular}
    \normalfont
  \end{center}
\item Player 1 has a better batting average than Player 2 in both the
  first and second half of the season, yet has a worse batting average
  overall
\item Consider the number of at-bats 
\end{itemize}
\end{frame}

\begin{frame}[fragile]\frametitle{Berkeley admissions data}
\begin{itemize}
\item The Berkeley admissions data is a well known data set regarding Simpsons paradox
\begin{verbatim}
?UCBAdmissions
data(UCBAdmissions)
     apply(UCBAdmissions, c(1, 2), sum)
          Gender
Admit      Male Female
  Admitted 1198    557
  Rejected 1493   1278
           .445   .304 <- Acceptance rate
\end{verbatim}
\end{itemize}
\end{frame}

\begin{frame}[fragile]
Acceptance rate by department
\begin{verbatim}
> apply(UCBAdmissions, 3, 
        function(x) c(x[1] / sum(x[1 : 2]), 
                      x[3] / sum(x[3 : 4])
                      )
        ) 
Dept  M    F
   A 0.62 0.82
   B 0.63 0.68
   C 0.37 0.34
   D 0.33 0.35
   E 0.28 0.24
   F 0.06 0.07
\end{verbatim}
\end{frame}


\begin{frame}[fragile]
Why? The application rates by department
\begin{verbatim}
> apply(UCBAdmissions, c(2, 3), sum)
        Dept
Gender     A   B   C   D   E   F
  Male   825 560 325 417 191 373
  Female 108  25 593 375 393 341
\end{verbatim}
\end{frame}

\begin{frame}\frametitle{Discussion}
\begin{itemize}
\item Mathematically, Simpson's paradox is not paradoxical
  \begin{eqnarray*}
    a / b & < & c / d\\
    e / f & < & g / h \\
    (a + e) / (b + f) & > & (c + g) / (d + h)
  \end{eqnarray*}
\item More statistically, it says that the apparent relationship between two
  variables can change in the light or absence of a third
\end{itemize}
\end{frame}

\section{Confounding}
\begin{frame}\frametitle{Confounding}
\begin{itemize}
\item Variables that are correlated with both the explanatory and response
  variables can distort the estimated effect
  \begin{itemize}
  \item Victim's race was correlated with defendant's race and death penalty
  \end{itemize}
\item One strategy to adjust for confounding variables is to {\bf stratify}
  by the confounder and then combine the strata-specific estimates
  \begin{itemize}
  \item Requires appropriately weighting the strata-specific estimates
  \end{itemize}
\item Unnecessary stratification reduces precision
\end{itemize}
\end{frame}

\section{Weighting}
\begin{frame}\frametitle{Aside: weighting}
\begin{itemize}
\item Suppose that you have two unbiased scales, one with variance $1$
  lb and and one with variance $9$ lbs
\item Confronted with weights from both scales, would you give both measurements
  equal creedance?
\item Suppose that $X_1 \sim N(\mu, \sigma_1^2)$ and $X_2 \sim N(\mu, \sigma_2^2)$
where $\sigma_1$ and $\sigma_2$ are both known
\item log-likelihood for $\mu$
  $$
  -(x_1 - \mu)^2 / 2\sigma_1^2 - (x_2 - \mu)^2 / 2\sigma_2^2
  $$
\end{itemize}
\end{frame}

\begin{frame}\frametitle{Continued}
\begin{itemize}
\item Derivative wrt $\mu$ set equal to 0
$$
(x_1 - \mu) / \sigma_1^2 + (x_2 - \mu) / \sigma_2^2 = 0 
$$
\item Answer $$\frac{x_1 r_1 + x_2r_2}{r_1 + r_2} = x_1 p + x_2 (1 - p)$$
where $r_i = 1/\sigma_i^2$ and $p = r_1 / (r_1 + r_2)$
\item Note, if $X_1$ has very low variance, its term dominates the estimate of $\mu$
\item General principle: instead of averaging over several unbiased estimates, take an average weighted according to inverse variances
\item For our example $\sigma_1^2 = 1$, $\sigma_2^2 = 9$ so $p = .9$
\end{itemize}
\end{frame}


\section{Mantel/Haenszel estimator}
\begin{frame}\frametitle{Mantel/Haenszel estimator}
\begin{itemize}
\item Let $n_{ijk}$ be entry $i$, $j$ of table $k$ 
\item  The $k^{th}$ sample odds ratio is $\hat \theta_k = \frac{n_{11k}n_{22k}}{n_{12k}n_{21k}}$
\item The Mantel Haenszel estimator is of the form $\hat \theta = \frac{\sum_k r_k \hat \theta_k}{\sum_k r_k}$
\item The weights are $r_k = \frac{n_{12k}n_{21k}}{n_{++k}}$
\item The estimator simplifies to $\hat \theta_{MH} = \frac{\sum_k n_{11k}n_{22k}/n_{++k}}{\sum_k n_{12k}n_{21k}/n_{++k}}$
\item SE of the log is given in Agresti (page 235) or Rosner (page 656)
\end{itemize}
\end{frame}

\begin{frame}[fragile]
\ttfamily\small
\begin{verbatim}
                          Center
     1      2      3      4      5      6     7      8
   S  F   S  F   S  F   S  F   S  F   S  F   S  F   S  F
T 11 25  16  4  14  5   2 14   6 11   1 10   1  4   4  2
C 10 27  22 10   7 12   1 16   0 12   0 10   1  8   6  1
n   73     52     38     33     29     21     14     13
\end{verbatim}
S - Success, F - failure \\
T - Active Drug,  C - placebo\footnote{Data from Agresti, Categorical Data Analysis, second edition}
\normalfont \\
$$
\hat \theta_{MH} = 
\frac{(11 \times 27) / 73 + (16 \times 10)/25 + \ldots + (4\times 1)/13}
{(10 \times 25) / 73 + (4 \times 22) / 25 + \ldots + (6 \times 2 ) / 13)} = 2.13
$$ \\
Also $\log\hat\theta_{MH} = .758$ and $\hat{SE}_{\log \hat\theta_{MH}} = .303$
\end{frame}

\begin{frame}\frametitle{CMH test}
\begin{itemize}
\item $H_0:\theta_1=\ldots=\theta_k = 1$ versus $H_a:\theta_1=\ldots=\theta_k \neq 1$
\item The CMH test applies to other alternatives, but is most powerful for the
  $H_a$ given above
\item Same as testing conditional independence of the response and exposure given the
  stratifying variable
\item CMH conditioned on the rows and columns for each of the $k$ contingency tables
  resulting in $k$ hypergeometric distributions and leaving only the $n_{11k}$ cells
  free
\end{itemize}
\end{frame}

\begin{frame}\frametitle{CMH test cont'd}
\begin{itemize}
\item Under the conditioning and under the null hypothesis 
  \begin{itemize} 
  \item $E(n_{11k})  = n_{1+k}n_{+1k}/n_{++k}$
  \item $\Var(n_{11k}) = n_{1+k}n_{2+k}n_{+1k}n_{+2k}/n_{++k}^2(n_{++k} - 1)$ 
  \end{itemize}
\item The CMH test statistic is 
  $$
  \frac{[\sum_k \{n_{11k} - E(n_{11k})\}]^2}{\sum_k \Var(n_{11k})}
  $$ 
\item For large sample sizes and under $H_0$, this test statistic is $\chi^2(1)$ (regardless
  of how many tables you are summing up)
\end{itemize}
\end{frame}

\begin{frame}[fragile]\frametitle{In R}
\begin{verbatim}
dat <- array(c(11, 10, 25, 27,  16, 22,  4, 10,
               14,  7,  5, 12,   2,  1, 14, 16,
                6,  0, 11, 12,   1,  0, 10, 10,
                1,  1,  4,  8,   4,  6,  2,  1),
             c(2, 2, 8))
mantelhaen.test(dat, correct = FALSE)
\end{verbatim}
Results: $CMH_{TS} = 6.38$ \\
P-value: $.012$ \\
Test presents evidence to suggest that the treatment and
response are not conditionally independent given center
\end{frame}

\begin{frame}\frametitle{Some final notes on CMH}
\begin{itemize}
\item It's possible to perform an analogous test in a random effects
  logit model that benefits from a complete model specification
\item It's also possible to test heterogeneity of the strata-specific
  odds ratios
\item Exact tests (guarantee the type I error rate) are also possible
  \texttt{exact = TRUE} in R
\end{itemize}
\end{frame}


\end{document}
